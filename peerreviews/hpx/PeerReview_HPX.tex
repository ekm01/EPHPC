%!TeX ts-program = xelatex
%!TeX encoding = utf-8 Unicode
\documentclass[ieeetran]{article}
\usepackage{amsmath, amssymb}
\usepackage{graphicx}
\thispagestyle{empty}

\begin{document}

\section*{Review of 'The HPX Parallel Programming Model' by Sören Weindel} % (fold)
\label{sec:review_of_openMP_an_introduction_by_lukas_limmer}

\subsection*{Summary} % (fold)
\label{sub:summary}
HPX is a C++ Standard library which enables us to develop and execute large-scale programs on diverse HPC systems in parallel.\ In that manner, HPX offers an alternative way of abstraction as well as intercommunication compared to traditional approaches such as MPI. It can also be seen as a unified solution for deficiencies that emerge when programming with different libraries and frameworks in combination.

\subsection*{Positive Aspects} % (fold)
\label{sub:positive_aspects}
The paper is concise and explanatory, as it covers complex concepts in a simple but highly comprehensive manner. It also follows a methodological rigor by thoroughly analyzing every aspect of the topic and validating them with several other findings or interpretations. This way, the reader can gain much deeper knowledge about the topic. Moreover, presenting the findings of various benchmarks makes the paper even more persuasive while giving an insight into the practical use of HPX at the same time.

\subsection*{Aspects to Improve} % (fold)
\label{sub:aspects_to_improve}
Last sentence of \textit{"Abstract"} about the contents of the paper is redundant, as it is described thoroughly at the end of \textit{"Introduction"}. There are also some sentences in the main part which are significantly long and contain complex information. Breaking them can help the reader get a finer grasp of the topic without losing the control.


\subsection*{Additional Comments} % (fold)
\label{sub:additional_comments}
There are only a few aspects to improve. Outside of that, the paper is overall well-organized and consists of remarkable findings of various references.


\subsection*{Overall Rating} % (fold)
\label{sub:overall_rating}
My overall rating for this paper would be 5 out of 5. 

\end{document}
