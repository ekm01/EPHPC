%!TeX ts-program = xelatex
%!TeX encoding = utf-8 Unicode
\documentclass[ieeetran]{article}
\usepackage{amsmath, amssymb}
\usepackage{graphicx}


\begin{document}

\section*{Review of 'OpenMP An Introduction' by Lukas Limmer} % (fold)
\label{sec:review_of_openMP_an_introduction_by_lukas_limmer}

\subsection*{Summary} % (fold)
\label{sub:summary}
OpenMP is an application programming interface that provides a wide variety of concepts for extending programming languages, such as C, to achieve shared memory parallelism on a system consisting of multiple computational units. The concepts of OpenMP are based on multi-threading, as the developer can create a number of threads, distribute the task among them and synchronize them by using the methods of OpenMP for consistent resource management.

\subsection*{Positive Aspects} % (fold)
\label{sub:positive_aspects}
Introduction part is very straightforward for a reader with little to no experience concerning the topic, which makes the paper more intriguing at the very first glance. Concepts explained in the main part are easy to understand thanks to the choice of words. Due to the well-organized structure of the paper, it is not difficult to understand the connections between the different chapters or paragraphs or to draw our own conclusions. Figures and code snippets, which are used to describe different concepts, are also clear and self-explanatory. Furthermore, the reader is provided with some valuable data as to performance comparison in the fourth chapter, as it gives a brief illustration of how a program is influenced by using those concepts.

\subsection*{Aspects to Improve} % (fold)
\label{sub:aspects_to_improve}
One of those aspects is punctuation. The lack of punctuation inbetween clauses mostly leads to confusing sentences, especially when the sentence is long. At the very beginning, there are repititions of some sentences (or semantically equivalent) which kind of breaks the flow in the paper. There are redundant sentences which are referring to previous chapters. An example for such sentence could be the first sentence of the chapter \textit{"Reduction"}. There is also a wrong use of the term \textit{"fork"} in page 2 in \textit{"Thread creation"}, as it is normally used for processes not threads. 

\subsection*{Additional Comments} % (fold)
\label{sub:additional_comments}
I personally think that the paper is very informative and well-structured. However, there are some small aspects and errors that are needed to be improved or corrected for the sake of the integrity of paper.

\subsection*{Overall rating} % (fold)
\label{sub:overall_rating}
My overall rating for this paper would be 4 out of 5.


\end{document}
