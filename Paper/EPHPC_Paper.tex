%!TeX ts-program = xelatex
%!TeX encoding = utf-8 Unicode
\documentclass[ieeetran]{article}
\usepackage{amsmath, amssymb}
\usepackage{graphicx}



\title{\includegraphics[width=0.43\textwidth]{tumlogo}\hspace{2ex}\includegraphics[width=0.25\textwidth]{maxlogo}\vspace{1ex}\\ \large \textbf{Max Planck Computing and Data Facility \\Chair of Computer Architecture and Parallel Systems} \vspace{10ex}\\ \huge Tensorflow \vspace{2ex}\\
\large Seminar: Efficient Programming of HPC Systems \\Frameworks and Algorithms\vspace{15ex}}


\author{Efe Kamasoglu}


\begin{document}

\maketitle

\pagebreak

\tableofcontents

\pagebreak

\addcontentsline{toc}{section}{Abstract}
\section*{Abstract}



\pagebreak


\section{Introduction} % (fold)
\label{sec:introduction}
TensorFlow is a free and open-source framework developed by Google Brain, which finds its application widely in the field of machine learning and artificial intelligence. It is used to build and train large-scale models according to the client's preferences and provided data sets. In order to train a model, TensorFlow carries out several computations by mapping them onto a variety of hardware, such as mobile devices or HPC systems consisting of multiple computational units with hundreds of GPUs.


\end{document}
